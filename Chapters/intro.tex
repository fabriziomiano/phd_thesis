\chapter*{Introduction}
\addcontentsline{toc}{chapter}{Introduction}
\markboth{}{Introduction}
\epigraph{\emph{The journey, not the destination matters.}}{Thomas S. Eliot}


One of the first connections between space-time symmetries and the conservation laws of physics was drawn by Emmy Noether. An informal statement of her theorem could be \emph{``If a system has a continuous symmetry property, then there are corresponding quantities whose values are conserved in time''}~\cite{thompson1994angular}, which means for example that the symmetry of the laws of physics under time translation result in the conservation of energy, or that a symmetry under spatial translation results in the conservation of momentum.

The Higgs boson discovery in $2012$ by the ATLAS and CMS collaborations was a milestone in the scope of particle physics~\cite{ATLASHiggs2012, CMSHiggs2012}. However, it did leave some unanswered questions. When attempting a unified description of the weak force and the electromagnetic force, whose intensities are enormously different, the interactions of the Higgs field with the so-called gauge bosons break the symmetry. It is known that the gauge bosons of the \ac{SM}, $\Wboson$ and $\Zboson$, acquire mass through their interactions with the Higgs field, but the photon does not, resulting in a huge difference in terms of interaction range: finite for $W$ and the $Z$, infinite for the photon. Kenneth Wilson in the early 1970’s noticed a problem that today is known as the \emph{hierarchy problem}: the Higgs boson, that gives mass to all fundamental particles, and to itself, happens to have a mass that is theoretically unstable and around a factor of $10^{16}$ larger than the electroweak scale. Such difference is considered ``unnatural''. Around the same time, a new symmetry was proposed: \ac{SUSY}. Such theory essentially extends the space-time symmetries into the quantum domain connecting classical quantities, such as space and time, to the spin of a particle. Most importantly, \ac{SUSY} gives rise to a particle that fits in with all the characteristics of a dark matter candidate, providing at the same time a natural solution to the hierarchy problem by cancelling the terms in the calculation of the Higgs boson mass arising from interactions of the Higgs boson with \ac{SM} particles.

The work presented in this thesis was carried out during a $3.5$-year PhD on the \acs{ATLAS} experiment at the \ac{LHC} within the scope of a third-generation \ac{SUSY} search: the search for the supersymmetric partner of the top quark in final states with jets and missing transverse momentum (\emph{$0$-lepton stop}). The results of this work were published in a paper in the Journal of High Energy Physics in September 2017~\cite{stop0L}. This analysis was carried out as part of the \acs{ATLAS} collaboration and the final result is a combination of the work carried out by the author and by other members of the collaboration. In Chapters~\ref{ch:theory},~\ref{ch:detector}, and~\ref{ch:evSimObjReco} a description of the theoretical framework, the experimental setup relevant for the scope in which this thesis is positioned, and the techniques used to reconstruct the physics objects needed to perform the analyses is given. Chapter~\ref{ch:trigger} contains the description of the \acs{ATLAS} trigger. Particular emphasis is given to the author’s contribution on the evaluation of the performance of the inner detector trigger. Chapters~\ref{ch:stop_ana} and~\ref{ch:bkgest} present the analysis carried out by the author as part of the \emph{$0$-lepton stop} analysis team within the \acs{ATLAS}-SUSY working group. The author contributed to the analysis effort providing an optimisation strategy of the regions in which the \ac{SUSY} signal was searched for, and a data-driven technique for the estimation of the irreducible $\ttZ (\to \nu \nu)$ background and its relative theory uncertainties. Chapter~\ref{ch:results} contains an overview of the statistical tools used to produce the results of this analysis. Appendix~\ref{app:ttzdm} presents the estimation of the irreducible $\ttZ (\to \nu \nu)$ background and its relative theory uncertainties for a Dark Matter search to which the author contributed. Finally, Appendix~\ref{app:sumbkgest} is a detailed summary of the selections employed for the background estimation described in Chapter~\ref{ch:bkgest}.