\chapter{The Standard Model, Supersymmetry, and the motivations behind it}
\label{ch:theory} \epigraph{\emph{A theory is something nobody believes, except the person who made it. An experiment is something everybody believes, except the person who made it.}} {Albert Einstein}

	The Standard Model (SM) of particle physics is an effective theory that aims to provide a general description of fundamental particles and the phenomena we see in nature, \ie\ the way they interact. Unfortunately, our understanding of nature is still limited due to some opened question to which the SM is not able to answer to, yet. 

	In this chapter, an overview of the SM will be presented in Section \ref{sec:SM} together with the limitations of such theory and some of the reasons behind the need of an extension. For the last decades theoretical physicsts have been trying to provide extensions to the SM, the so-called beyond-the-SM theories. Among these, one of the most popular, Supersymmetry which will be discussed in Section \ref{sec:SUSY}.  

	
	\section{The Standard Model}
	\label{sec:SM}

		The 20$^{th}$ century can be considered a quantum revolution. Several experiments led to discoveries which were found to be, together with the formalised theory, a solid base of the Standard Model of particle physics and our description of nature. Several particles were predicted first by the SM and then experimentally observed \eg\ the \Wboson\ and the \Zboson\ bosons, the $\tau$ lepton, \cite{Herrero1998}, and more recently the Higgs boson at the LHC discovered by ATLAS \cite{ATLASHiggs2012} and CMS \cite{CMSHiggs2012}.

		The SM lays on a Quantum Field Theory (QFT) where particles are treated like fields. It can describe three of the four fundamental forces; weak, electromagnetic, and strong. As of today, gravity is not considered in the SM. Sections \ref{sec:SMov} and \ref{sec:SMlim} will be focused on the description of the fields together with the carriers of the information, and on the limitations that such theory implies, respectively.


		\subsection{Overview}
		\label{sec:SMov}

			The most general classification of the particles within the SM can be made by means of spin. Fermions have half-integer spin values - and are usually referred to as matter -, and bosons have integer-spin values. A noteworthy subset of bosons is formed by the Spin-1 bosons (also known as gauge bosons). These can be considered the information carriers or, in fact, the mediators of the forces. 


			\subsection*{Fermions}

				Six quarks and six leptons belong to the fermions family. The six quarks of the SM can be grouped into doublets of three generations. There are three generations, each containing four particles; one charged lepton, one neutrino, and one up- and down-type quark. The masses of the charged leptons and quarks increase with the generation. 

				\begin{equation*}
				\label{eqn:quark_doublets}
					\begin{pmatrix} u \\ d \end{pmatrix}, \qquad 
					\begin{pmatrix} c \\ s \end{pmatrix}, \qquad 
					\begin{pmatrix} t \\ b \end{pmatrix}
				\end{equation*}

				%\noindent Generations are classified according to the state of chirality, which is a non-physical concept close enough to helicity which, in turn, is the projection of the spin onto the direction of momentum. Chirality is equivalent to helicity for massless particles and, as per helicity states, there can either be left-handed chiral states or right-handed chiral states.
				The up-type quarks (up, charm, top) have charge $+\frac{2}{3}e$ and the down-type quarks (down, strange, beauty/bottom) have charge $-\frac{1}{3}e$, where $e$ is the electron charge. Quarks also have another quantum number that can be seen as the analogue of the electric charge, the colour charge. There are three different states of the colour charge; ``red'', ``green'' and ``blue''. Three of the six leptons ($e\,,\mu\,,\tau$) are electrically charged and three of them electrically neutral ($\nu_e\,, \nu_{\mu}\,, \nu_{\tau})$ and, as quarks they can also grouped into generations:

				\begin{equation*}
				\label{eqn:lepton_flavor_doublets}
					\begin{pmatrix} \nu_e      \\ e^-    \end{pmatrix}, \qquad
					\begin{pmatrix} \nu_{\mu}  \\ \mu^-  \end{pmatrix}, \qquad
					\begin{pmatrix} \nu_{\tau} \\ \tau^- \end{pmatrix}
				\end{equation*}


				



			\subsection*{Forces of Nature}	



		\subsection{Limitations of the Standard Model}
		\label{sec:SMlim}





	\section{Supersymmetry}
	\label{sec:SUSY}

		\subsection{Why SUSY?}

		\subsection{Minimal Supersymmetric Standard Model}

		\subsection{\emph{R}-parity SUSY}
		
		\subsection{Simplified models}

		\subsection{Phenomenological MSSM}
