\chapter{Search for top squarks in all-hadronic final states}
\label{ch:stop_ana}
\epigraph{\emph{In God we trust. All others must bring data.}}{William E. Deming}

	In this chapter the core of this thesis will be presented, namely the search for the direct pair-production of the supersymmetric partner of the top quark in all-hadronic final states using $36.1\, \ifb$ of \pp\ collisions, at a centre-of-mass energy $\rts = 13$ \TeV, delivered by the \ac{LHC} and collected by the \ac{ATLAS} detector. The results produced were published in a paper in the Journal of High Energy Physics in December 2017~\cite{stop0L}. A previous version of the analysis was also made public, using $13.3\,\ifb$ collected at $\rts = 13$ \TeV, with an earlier subset of the whole $2015+2016$ dataset, documented in an ATLAS conference note~\cite{ICHEPstop0L}. Although both versions contain author's contributions, only the results of the most recent analysis will be hereby discussed, as it represents the most updated, improved and extended version. Specifically, the optimisation of the search strategy, as well as the estimates of the number of events in the search regions for one of the most important backgrounds, and the evaluation of the related theory uncertainties, characterised the author's contributions.

	This chapter will be structured as it follows: an excursus on the simplified \ac{SUSY} models considered will be presented in Section~\ref{sec:susysig}; the selection of the events and the objects used in both data and \ac{MC} will be presented in Section~\ref{sec:evtsel}; the variables used and the optimisation of the regions in which the \ac{SUSY} signals were searched for will be presented in Section~\ref{sec:SRs}; the nominal procedure used  for the background estimation will be discussed in Section~\ref{sec:bkgest}, with particular focus on the data-driven background estimation in Section~\ref{sec:ddbkgest}; the results, together with their interpretation, will finally be presented in Section~\ref{sec:results}.


	\section{SUSY Signals}
	\label{sec:susysig}

		As already introduced in Section~\ref{sec:SUSYPheno} when discussing the phenomenology of the top squark, the signals considered in this work are generated using simplified models 

		\subsection{Benchmark processes}

		\subsection{\ac{MC} samples}


	\section{Event Selection}
	\label{sec:evtsel}

		\subsection*{Baseline Object Selection}

			\subsubsection*{Leptons}
			
			\subsubsection*{Photons}
			
			\subsubsection*{Jets}


		\subsection*{Overlap Removal}

		\subsection*{Signal Object Selection}

			\subsubsection*{Leptons}

			\subsubsection*{Photons}

			\subsubsection*{Jets}


	\section{Signal Regions}
	\label{sec:SRs}

		\subsection{Variables used}

		\subsection{Optimisation}


	\section{Nominal Background Estimation}
	\label{sec:bkgest}

		\subsection{Control Regions}

		\subsection{Validation Regions}


	\section{Data-Driven Background Estimation}
	\label{sec:ddbkgest}


	\section{Results and Interpretation}
	\label{sec:results}








