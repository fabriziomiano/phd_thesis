\chapter*{Conclusions}
\addcontentsline{toc}{chapter}{Conclusions}
\markboth{}{Conclusions}
\epigraph{\emph{Every new beginning comes from some other beginning's end.}} {Seneca}


The main outcome presented in this thesis is the best results to date of the search for the supersymmetric partner of the top quark in all-hadronic final states using the full $36.1\, \ifb$ dataset $(2015+2016)$ of \pp\ collisions at a centre-of-mass energy $\rts = 13$ \TeV\ delivered by the \ac{LHC} and collected by the \ac{ATLAS} detector~\cite{stop0L}. The $\ttZ (\to \nu \nu)$ irreducible \ac{SM} background and the relative theory uncertainties were estimated using a data-driven \ttgamma\ \ac{CR} and scenarios in which $R$-parity is conserved were targeted and final states with high-\pt\ jets and large missing transverse momentum, were addressed using selection criteria optimised accordingly. No significant deviation between the expected Standard Model events and the data was found.

A statistical interpretation of the results was carried out in order to set 95\% \ac{CL} exclusion limits on the parameters of the models considered resulting in the exclusion of top-squark masses in the range $450-1000$ \GeV\ for \ninoone\ masses below $160$ \GeV\ improving the Run-1 results~\cite{stop0LRun1} by almost $400$ \GeV. A new \ac{SR} was designed to address the diagonal case, $m_{\stopone} \sim m_t + m_{\ninoone}$, where top-squark masses in the range $235-590$ \GeV\ are excluded. Additionally, limits that take into account an additional decay of $\stopone \to b \chinoonepm$ were set excluding top-squark masses between $450$ and $850$ \GeV\ for neutralino masses below $240$ \GeV\ and $B(\stopone \to t \ninoone) = 50\%$ for $m_{\chinopm} = m_{\ninoone} + 1$ \GeV. Limits were also derived in two pMSSM models, where one model assumes a wino-like \ac{NLSP} and the other model is constrained by the dark-matter relic density. In addition, limits were set in terms of one \ac{pMSSM}-inspired simplified model where $m_{\chinoonepm} = m_{\ninoone} + 5$ \GeV\ and $m_{\ninotwo} = m_{\ninoone} + 10$ \GeV. Gluino masses were constrained to be above $1800$ \GeV\ for \stopone\ masses below $800$ \GeV\ for gluino-mediated top squark production where $m_{\stopone} = m_{\ninoone} + 5$ \GeV. 

The data-driven estimation of the irreducible $\ttZ (\to \nu \nu)$ background and its relative theory uncertainties were also used in another ATLAS search, for dark matter produced in association with third-generation quarks~\cite{DMhf}, of which a brief overview was presented. Here, mediator masses between $10$ and $50$ \GeV\ for scalar mediators, assuming couplings equal to unity and a dark-matter mass of $1$ \GeV, were excluded at 95\% CL. Finally, even though the analysis was expected to be sensitive to models with pseudoscalar mediators with masses between $10$ and $100$ \GeV, limits could not be set for this model for the coupling assumption of $g = 1.0$ due to a small excess in the observed data.	