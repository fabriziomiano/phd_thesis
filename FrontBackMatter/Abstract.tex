% % Abstract
% \pagestyle{plain} % Suppress headers for the pre-content pages

% %\renewcommand{\abstractname}{Abstract} % Uncomment to change the name of the abstract
\thispagestyle{empty}
\pdfbookmark[0]{Abstract}{Abstract} % Bookmark name visible in a PDF viewer

% \begingroup
% \let\clearpage\relax
% \let\cleardoublepage\relax
% \let\cleardoublepage\relax

% \chapter*{Abstract}
% This thesis presents searches for supersymmetry in $\sqrt{s}=13$ TeV proton-proton collisions at the LHC using data collected by the ATLAS detector in 2015, 2016 and 2017. Events with 4 or more jets and missing transverse energy were selected. Kinematic variables were investigated and optimisations were performed to increase the sensitivity to supersymmetric signals. Standard Model backgrounds were estimated by means of Monte Carlo simulations and data-driven techniques. Before analysing the data in the blinded signal regions the agreement between data and background predictions and the extrapolations from control and validation regions to signal regions were validated. The analysis yielded no significant excess in any of the analyses performed. Therefore limits were set and the results were interpreted as lower bounds on the masses of supersymmetric particles in various scenarios and models.

% \endgroup			

% \vfill
        \begin{center}
        	\bigskip
            {\normalsize \href{http://www.sussex.ac.uk/}{\myUni} \par} % University name in capitals
            {\normalsize \myFaculty \par} % Faculty name
            {\normalsize \myDepartment \par} % Department name
            \bigskip\vspace*{.02\textheight}
            %{\normalsize \myDegree\par} % Degree name
            {\Large Doctoral Thesis}\par
            \bigskip%\vspace*{.02\textheight}
            
            {\rule{\linewidth}{1pt}\\%[0.4cm]
            \Large \myTitle \par} % Thesis title
            \rule{\linewidth}{1pt}\\[0.4cm]
            
            \bigskip
			{\normalsize by \myName \par} % Author name
            \bigskip\vspace*{.06\textheight}
            {\Huge\textit{Abstract} \par}
            \bigskip

        \end{center}

        \noindent This thesis presents searches for supersymmetry in $\sqrt{s}=13$ TeV proton-proton collisions at the LHC using data collected by the ATLAS detector in 2015, 2016 and 2017. Events with 4 or more jets and missing transverse energy were selected. Kinematic variables were investigated and optimisations were performed to increase the sensitivity to supersymmetric signals. Standard Model backgrounds were estimated by means of Monte Carlo simulations and data-driven techniques. Before analysing the data in the blinded signal regions the agreement between data and background predictions and the extrapolations from control and validation regions to signal regions were validated. The analysis yielded no significant excess in any of the analyses performed. Therefore limits were set and the results were interpreted as lower bounds on the masses of supersymmetric particles in various scenarios and models.
