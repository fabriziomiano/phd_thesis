% % Abstract

\thispagestyle{empty}
\pdfbookmark[0]{Abstract}{Abstract} % Bookmark name visible in a PDF viewer

\begin{center}
%	\bigskip

    {\normalsize \href{http://www.sussex.ac.uk/}{\myUni} \\} % University name in capitals
    {\normalsize \myFaculty \\} % Faculty name
    {\normalsize \myDepartment \\} % Department name
    \bigskip\vspace*{.02\textheight}
    {\Large \textsc{Doctoral Thesis}}\par
    \bigskip
    
    {\rule{\linewidth}{1pt}\\%[0.4cm]
    \Large \myTitle \par} % Thesis title
    \rule{\linewidth}{1pt}\\[0.4cm]
    
    \bigskip
	{\normalsize by \myName \par} % Author name
    \bigskip\vspace*{.06\textheight}
\end{center}

    {\centering\Huge\textsc{\textbf{Abstract}} \par}
    \bigskip



    \noindent This thesis presents the search for the supersymmetric partner of the top quark in $\sqrt{s}=13$ \TeV\ proton-proton collisions at the LHC using data collected by the ATLAS detector in 2015 and 2016. Results were interpreted considering natural supersymmetric extensions of the Standard Model in $R$-parity conserving decays. Events characterised by four or more jets and missing transverse momentum in the final states were selected. The performance of the tracking algorithms used by the ATLAS online trigger were studied. Optimisation studies of the search regions to increase the sensitivity to supersymmetric signals were performed and data-driven techniques to estimate Standard Model backgrounds were employed. The agreement between data and background predictions was extensively checked and the extrapolations from background-enriched regions to signal-enriched regions were validated. The analysis yielded no significant excess therefore exclusion limits on various models were set.


